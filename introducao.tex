% Workaround para criação de capítulo não-numerado alinhado à margem
\chapter*{}
\noindent
\phantomsection{\MakeUppercase{\textbf{Introdução}}}
\addcontentsline{toc}{chapter}{INTRODUÇÃO}

% \newline
% \newline

%\section{Motivação}

\vspace{1.4cm}

Os setores da agropecuária e da mineração têm sido pilares fundamentais na história econômica do Brasil, contribuindo para o desenvolvimento tanto no âmbito regional quanto nacional. Em 2021, com as commodities em alta em meio à pandemia, a soma de agro e mineração superou a de manufatura no PIB brasileiro pela primeira vez em décadas e a tendência se manteve em 2022 com os efeitos da guerra da Ucrânia \cite{BBC2023}.

Sendo assim, com a crescente demanda por recursos minerais essenciais, como argila, calcário e ferro, a mineração não apenas fornece matérias-primas para diversas indústrias, mas também gera empregos e contribui para a economia do Brasil. Com isso, o setor de mineração volta ao foco como uma das atividades econômicas importantes do Brasil, contribuindo significativamente para o PIB nacional e para a geração de empregos \cite{IBRAM2020}.

No entanto, a atividade mineradora também traz consigo desafios e riscos significativos, especialmente no que diz respeito ao gerenciamento de resíduos e de rejeitos resultantes dos processos de beneficiamento do minério, devido aos seus impactos ambientais e sociais \cite{ipea2017boletim}.

A gestão de rejeitos de mineração é uma questão crítica que requer abordagens inovadoras e tecnologias avançadas para garantir a segurança e a sustentabilidade ambiental das operações \cite{EPA2019}. Pois a má gestão dos rejeitos pode levar a uma série de consequências adversas, incluindo mudanças ambientais, desvalorização de imóveis e até mesmo desastres como os que ocorreram em Mariana e Brumadinho \cite{ipea2017boletim}, que serviram como um chamado de alerta para a necessidade urgente de melhorar os sistemas de segurança e prevenção em barragens de rejeitos.

Nesse contexto, a engenharia da computação oferece uma variedade de soluções inovadoras que podem ser aplicadas para aumentar a segurança e reduzir os riscos associados às barragens de rejeitos. A prevenção de desastres em barragens de rejeitos requer a implementação de tecnologias modernas e sistemas de monitoramento sofisticados para identificar e mitigar os riscos potenciais \cite{ICMM2021}.
    
\vspace{0.5cm}
\textbf{Objetivo}
\vspace{0.5cm}

Este trabalho teve como objetivo utilizar como exemplo um higrômetro de resistência para solo, implementado por meio da plataforma Arduino, de forma a representar as diversas instrumentações geotécnicas como uma ferramenta na detecção precoce de problemas que pudessem comprometer a segurança de uma barragem de rejeito para gerar alertas de forma a prevenir desastres. O sistema proposto foi desenvolvido utilizando a plataforma open-source Arduino, equipada com sensores higrométricos de alta precisão, proporcionando assim um monitoramento contínuo e de baixo custo das condições de umidade do solo nas estruturas da barragem. A arquitetura do sistema contempla, adicionalmente, um módulo de comunicação para transmissão automática dos dados coletados, bem como a integração com as bases de dados meteorológicos disponibilizadas pelo Centro Nacional de Monitoramento e Alertas de Desastres Naturais (CEMADEN) e pelo Instituto Nacional de Meteorologia (INMET). Esta integração permite correlacionar os níveis de umidade detectados in loco com parâmetros climatológicos críticos, como precipitação acumulada e previsões pluviométricas para a região monitorada.

Ao integrar tecnologias de engenharia da computação com instrumentação geotécnica é possível desenvolver um sistema de segurança robusto e eficiente que possa ajudar a prevenir desastres semelhantes aos de Mariana e Brumadinho. O sistema implementado possibilita a emissão automatizada de alertas graduais de segurança para autoridades competentes e população potencialmente afetada, estabelecendo diferentes níveis de risco baseados na análise integrada dos dados higrométricos e meteorológicos. Esta pesquisa tem o potencial de contribuir significativamente para a segurança das operações de mineração e para a proteção do meio ambiente e das comunidades afetadas por essas atividades, oferecendo uma solução tecnologicamente acessível e metodologicamente consistente para o monitoramento preventivo em barragens de rejeitos.

\vspace{0.5cm}
\textbf{Estrutura do Trabalho}
\vspace{0.5cm}

No capítulo 1 foi realizada uma pesquisa bibliográfica sobre a atividade de mineração, o PIB nacional, a geração de empregos e a produção de resíduos. Ainda nesse capítulo foi abordado os tipos de barragens de rejeitos, as legislações vigentes, um breve resumo sobre os acidentes de Mariana e Brumadinho e os sistemas de alerta existentes. No capítulo 2 foi descrito o objetivo geral e os objetivos específicos relacionando as medições com a análise de dados e tomada de decisões, assim como a notificação de sistemas de alerta e as APIs. No capítulo 3 foi realizada uma análise dos materiais e métodos, fazendo uso de diagramas, componentes do Arduino e obtenção de dados externos. No capítulo 4 foi apresentado os resultados e conclusões com base nos materiais e métodos utilizados e no capítulo 5 foi apresentada a conclusão final do trabalho.

\vspace{10mm} %5mm vertical space

