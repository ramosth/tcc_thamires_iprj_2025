%% Elemento obrigatório (Figura 15).
%% Consiste em uma tradução do resumo em português para uma
%% língua estrangeira (em inglês, ABSTRACT; em espanhol, RESUMEN;
%% em francês, RÉSUMÉ), em um único parágrafo, seguido das palavras-
%% -chave representativas do conteúdo do trabalho, na língua estrangeira
%% escolhida.
%% O resumo em outra língua também é precedido pela referência
%% do trabalho, substituindo-se o título em português pelo título na língua
%% estrangeira adotada.
%% No caso de teses, é possível incluir dois resumos em língua es-
%% trangeira.
%% A apresentação gráfica e a ordem dos elementos seguem a mes-
%% ma orientação do resumo em português.

\begin{resumo}[Abstract]
\begin{otherlanguage*}{english}
\begin{SingleSpace}

\noindent
\begin{flushleft}
\entradaAutor{}. \textit{\englishTitle{}} 2024.\pageref{LastPage} f. Trabalho de Conclusão de Curso (Graduação em Engenharia de Computação) - Instituto Politécnico, Universidade do Estado do Rio de Janeiro, Nova Friburgo, 2024.
\end{flushleft}
\vspace{\onelineskip}

\setlength{\parindent}{1.3cm}

Mining plays an essential role in the Brazilian economy, contributing to regional and national development, as well as generating jobs and boosting GDP. However, inadequate management of the tailings resulting from these activities can lead to serious environmental and social impacts, as demonstrated by the disasters in Mariana and Brumadinho. Given this scenario, there is an urgent need to improve safety and prevention systems for tailings dams. Computer engineering offers innovative solutions to meet this challenge, through the implementation of advanced technologies and sophisticated monitoring systems. This paper investigates the application of monitoring systems based on resistance hygrometers for disaster prevention in mining tailings dams, a critical problem highlighted by the catastrophic incidents in Mariana and Brumadinho. The methodology is based on the integration of Arduino devices equipped with high-precision hygrometric sensors and advanced geotechnical instrumentation, establishing a continuous monitoring network for the early detection of changes in soil moisture that could compromise the structural stability of dams. The system developed incorporates a data processing layer that combines readings from local sensors with meteorological information obtained in real time via APIs from the CEMADEN and INMET portals, making it possible to correlate data on accumulated precipitation, rainfall forecasts and soil saturation. Preliminary results show that the proposed platform detects critical changes in humidity patterns 72 to 96 hours in advance, making it possible to automatically issue multiple levels of safety alerts to authorities and the potentially affected population. It is concluded that the technological solution presented constitutes an economically viable and technically robust approach to significantly increase safety in tailings dams, contributing to safer and more sustainable mining operations. The research aims to make a significant contribution to the safety of mining operations and the protection of the environment and affected communities by offering an innovative and technologically advanced approach to mitigating the risks associated with tailings dams.

\vspace{\onelineskip}
\noindent Keywords: Tailings dams; Arduino; Hygrometers; Environmental monitoring; Warning systems; Brumadinho; Environmental accident; Environmental risk. 

\end{SingleSpace}
\end{otherlanguage*}
\end{resumo}