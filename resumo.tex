%% 3.1.9 Resumo em língua portuguesa
%% Elemento obrigatório (Figura 14).
%% Consiste na apresentação sucinta dos pontos relevantes do texto,
%% em um único parágrafo. O resumo deve conter entre 150 e 500 pala-
%% vras e fornecer uma visão rápida e clara dos objetivos, da metodologia,
%% dos resultados e das conclusões do trabalho. Na elaboração do resumo,
%% deve-se usar o verbo na voz ativa, na terceira pessoa do singular.

%% Fonte -> TNR ou Arial, corpo 12.
%% A palavra RESUMO deve aparecer em letras maiúsculas
%% e em negrito.
%% O uso de itálico é permitido em palavras estrangeiras.
%% O uso de letras maiúsculas nas palavras-chave
%% restringe-se ao início da palavra, em nomes próprios
%% e siglas, se for o caso.

%% Alinhamento -> A palavra RESUMO deve estar localizada na margem
%% superior da folha e centralizada, e a referência, alinhada
%% à margem esquerda;
%% O alinhamento é justificado para o texto do resumo,
%% que inicia com parágrafo, e para as palavras-chave.

%% Espaçamento -> A palavra RESUMO deve ser separada da referência por
%% duas linhas em branco de 1,5;
%% Espaço 1 na referência e no resumo e, nas palavras-
%% chave, espaço 1,5.

%% Formato do papel,
%% orientação e margens -> Conforme especificado na seção 1.1.

%% Pontuação -> As palavras-chave devem ser separadas por ponto e
%% terminadas por ponto.

\begin{resumo}
\begin{SingleSpace}

\noindent
\begin{flushleft}
\entradaAutor{}. \textit{\imprimirtitulo}. \the\year. \pageref{LastPage} f. Trabalho de Conclusão de Curso (Graduação em Engenharia de Computação) - Instituto Politécnico, Universidade do Estado do Rio de Janeiro, Nova Friburgo, \the\year.
\end{flushleft}
\vspace{\onelineskip}

\setlength{\parindent}{1.3cm}

A mineração desempenha um papel essencial na economia brasileira, contribuindo para o desenvolvimento regional e nacional, além de gerar empregos e impulsionar o PIB. No entanto, a gestão inadequada dos rejeitos resultantes dessas atividades pode acarretar em graves impactos ambientais e sociais, como demonstrado pelos desastres em Mariana e Brumadinho. Diante desse cenário, torna-se urgente aprimorar os sistemas de segurança e prevenção em barragens de rejeitos. A engenharia da computação oferece soluções inovadoras para enfrentar esse desafio, por meio da implementação de tecnologias avançadas e sistemas de monitoramento sofisticados. Este trabalho investiga a aplicação de sistemas de monitoramento baseados em higrômetros de resistência para prevenção de desastres em barragens de rejeitos de mineração, problema crítico evidenciado pelos incidentes catastróficos ocorridos em Mariana e Brumadinho. A metodologia fundamenta-se na integração de dispositivos Arduino equipados com sensores higrométricos de alta precisão e instrumentação geotécnica avançada, estabelecendo uma rede de monitoramento contínuo para detecção precoce de alterações na umidade do solo que possam comprometer a estabilidade estrutural das barragens. O sistema desenvolvido incorpora uma camada de processamento de dados que combina leituras dos sensores locais com informações meteorológicas obtidas em tempo real através de APIs dos portais CEMADEN e INMET, permitindo correlacionar dados de precipitação acumulada, previsões pluviométricas e saturação do solo. Os resultados preliminares demonstram que a plataforma proposta detecta com antecedência de 72 a 96 horas alterações críticas nos padrões de umidade, possibilitando a emissão automatizada de múltiplos níveis de alertas de segurança para autoridades e população potencialmente afetada. Conclui-se que a solução tecnológica apresentada constitui uma abordagem economicamente viável e tecnicamente robusta para incrementar significativamente a segurança em barragens de rejeitos, contribuindo para operações de mineração mais seguras e sustentáveis. A pesquisa visa contribuir significativamente para a segurança das operações de mineração e para a proteção do meio ambiente e das comunidades afetadas, oferecendo uma abordagem inovadora e tecnologicamente avançada para mitigar os riscos associados às barragens de rejeitos.

\vspace{\onelineskip}
\noindent Palavras-chave:  Barragens de rejeitos; Arduino; Higrômetros; Monitoramento ambiental; Sistemas de alerta; Brumadinho; Acidente ambiental; Risco ambiental.

\end{SingleSpace}
\end{resumo}