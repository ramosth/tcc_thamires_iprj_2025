\chapter*{}
\noindent
\phantomsection{\MakeUppercase{\textbf{Conclusão e Trabalhos futuros}}}
\addcontentsline{toc}{chapter}{CONCLUSÃO E TRABALHOS FUTUROS}
\vspace{1cm}

Aqui vai a conclusão do trabalho, recapitulando o que foi desenvolvido e os benefícios promovidos pelo que foi desenvolvido no trabalho. 

Em seguida, falar sobre possíveis trabalhos futuros ou continuações do trabalho.

\begin{quote}
\textit{A conclusão proporciona um resumo sintético, mas completo, da argumentação,
das provas consignadas no desenvolvimento do trabalho, como decorrência
natural do que já foi demonstrado. Essa parte deve reunir as características do que
chamamos de síntese interpretativa dos argumentos ou dos elementos contidos no
desenvolvimento do trabalho.}
\end{quote}

\vspace{0.5cm}
\textit{Mais detalhes sobre as regras de construção de uma tese, ver o documento:
roteiro\_uerj\_web.pdf}